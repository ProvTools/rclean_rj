% !TeX root = RJwrapper.tex
\title{Rlean for easier code editing}
\author{by Author One, Author Two and Author Three}

\maketitle

\abstract{

R has enabled the rapid production of scientific software without
languistically imposed restrictions of many software best practices
for accessible code. With the goal of making the task of simplifying
scripts easier, we have created the \texttt{Rclean} package, which
uses recent advances in language specific data provenance capture to
enable a user to isolate the essential code, data and dependencies
needed to produce a specific result. As even relatively short scripts
can quickly become computationally complicated and opaque, we predict
that the code simplification method provided by \texttt{Rclean} and
similar tools will have significant impact on the transparency and
reproducibility of R code. 

}

\section{Outline}


\begin{itemize}
\item Background: R has enabled the rapid production of scientific
  software without the imposed restrictions of software best practices
  for concision, clarity or general transparency and reproducibility.
\item Define code cleaning goals
\item Define data provenance and describe how it enables code cleaning
\item Introduce Rclean, basics of API
\item Example: micro.R and/or messycode.R
\item Conclusion: Rclean is just begninning to crack the surface of
  provenance aware tools.
\end{itemize}


\section{Section title in sentence case}

Introductory section which may include references in parentheses
\citep{R}, or cite a reference such as \citet{R} in the text.

\section{Another section}

This section may contain a figure such as Figure~\ref{figure:rlogo}.

\begin{figure}[htbp]
  \centering
  \includegraphics{Rlogo-5}
  \caption{The logo of R.}
  \label{figure:rlogo}
\end{figure}

\section{Another section}

There will likely be several sections, perhaps including code snippets, such as:

\begin{example}
  x <- 1:10
  result <- myFunction(x)
\end{example}

\section{Summary}

This file is only a basic article template. For full details of \emph{The R Journal} style and information on how to prepare your article for submission, see the \href{https://journal.r-project.org/share/author-guide.pdf}{Instructions for Authors}.

\bibliography{RJreferences}

\address{Author One\\
  Affiliation\\
  Address\\
  Country\\
  (ORCiD if desired)\\
  \email{author1@work}}

\address{Author Two\\
  Affiliation\\
  Address\\
  Country\\
  (ORCiD if desired)\\
  \email{author2@work}}

\address{Author Three\\
  Affiliation\\
  Address\\
  Country\\
  (ORCiD if desired)\\
  \email{author3@work}}
