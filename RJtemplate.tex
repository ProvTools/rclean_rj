% !TeX root = RJwrapper.tex
\title{Rlean for easier code editing}
\author{by Author One, Author Two and Author Three}

\maketitle

\abstract{

R has enabled the rapid production of scientific software without
languistically imposed restrictions of many software best practices
for accessible code. With the goal of making the task of simplifying
scripts easier, we have created the \texttt{Rclean} package, which
uses recent advances in language specific data provenance capture to
enable a user to isolate the essential code, data and dependencies
needed to produce a specific result. As even relatively short scripts
can quickly become computationally complicated and opaque, we predict
that the code simplification method provided by \texttt{Rclean} and
similar tools will have significant impact on the transparency and
reproducibility of R code. 

}

\section{Outline}


\begin{itemize}
\item Background: R has enabled the rapid production of scientific
  software without the imposed restrictions of software best practices
  for concision, clarity or general transparency and reproducibility.
\item Define code cleaning goals
\item Define data provenance and describe how it enables code cleaning
\item Introduce Rclean, basics of API
\item Example: micro.R and/or messycode.R
\item Conclusion: Rclean is just begninning to crack the surface of
  provenance aware tools.
\end{itemize}



\section{The need for cleaner code}

R has enabled the rapid production of scientific software without the
imposed restrictions of software best practices for concision, clarity
or general transparency and reproducibility. This is good in that in
provides a much lower entry into software creation for scientists
without formal training in computer science or software
engineering. It also creates the potential for computer processes that
are impossible to openly share, such as analyses with ``black-box''
propreitary software, or tedious to document (e.g. spreadsheet
operations for data processing). However, it also has created the
potential to generate incorrect and/or obtuse code that also produces
incorrect and obtuse results that are difficult to validate or
replicate. 

For the researcher, a significant part of the difficulty in creating
transparent code is that the rapid, informal practices lead to the
generation of lots of results, many of which are not used in the final
product. This leads to bloated code that is more susceptible to
``code-rot'' and is more complex and less transparent in the context
of future use. Enabling R programmers, especially those involved in
the sciences, to more easily examine and extract the essential code
from their coding efforts would help reduce the complexity of a
researcher's code. A non-trivial barrier to this is that many
scientists have little time or incentive to take time to meticulously
pick back through their code to extract results by hand, especially
when their code is already complicated. Producing an automated means
for a user to hone their code by extracting the essential elements
could lower the barrier to creating simpler ``cleaner'' code.

One challenge to this is that, in a general case, until code is
executed, it cannot always be determined what code produces an
output. For example, two functions with the same name but from
different libraries could be used to create an output. Also, control
statements (e.g. \emph{if} and \emph{else}) can change what code is
executed. The solution to this callenge is tracking the relationships
among functions and data such that a record of the pathways from
inputs to a result are known. Thankfully, the field of ``data
provenance'' has produced a rigorous, formalized way to do exactly
this and recent efforts have begun to produce data provenance capture
tools in a variety of contexts and languages \cite{Pasquier2017}.

To address this need for creating clearer code and increase the
reproducibility of research, we have created the \texttt{Rclean}
package \cite{Rclean}. \texttt{Rclean} uses data provenance captured
for a given script to extract the minimal code needed to generate a
result. Here we use result in a very general sense that includes:
files (e.g. figures and tables saved to disk), objects in the script
or errors and warnings produced upon execution. Here, we detail the
mechanics of the \texttt{Rclean} package, struture of the user and
walk through a short example of its application. 

\section{Data provenance capture and analysis for code cleaning}

\section{The Rclean API}


\section{Example: micro.R and/or messycode.R}


\section{Conclusion }

Rclean is just begninning to crack the surface of
  provenance aware tools

\section{Section title in sentence case}

Introductory section which may include references in parentheses
\citep{R}, or cite a reference such as \citet{R} in the text.

\section{Another section}

This section may contain a figure such as Figure~\ref{figure:rlogo}.

\begin{figure}[htbp]
  \centering
  \includegraphics{Rlogo-5}
  \caption{The logo of R.}
  \label{figure:rlogo}
\end{figure}

\section{Another section}

There will likely be several sections, perhaps including code snippets, such as:

\begin{example}
  x <- 1:10
  result <- myFunction(x)
\end{example}

\section{Summary}

This file is only a basic article template. For full details of \emph{The R Journal} style and information on how to prepare your article for submission, see the \href{https://journal.r-project.org/share/author-guide.pdf}{Instructions for Authors}.

\bibliography{RJreferences}

\address{Author One\\
  Affiliation\\
  Address\\
  Country\\
  (ORCiD if desired)\\
  \email{author1@work}}

\address{Author Two\\
  Affiliation\\
  Address\\
  Country\\
  (ORCiD if desired)\\
  \email{author2@work}}

\address{Author Three\\
  Affiliation\\
  Address\\
  Country\\
  (ORCiD if desired)\\
  \email{author3@work}}
